\documentclass[a4paper,10pt]{article}
\usepackage[T1]{fontenc}
\usepackage[utf8]{inputenc}
\usepackage{float}
\usepackage{tikz}

\begin{document}

\title{Resultados Tarea 5}

\author{María Laura Pérez Lara}

\maketitle
Esta es una de las primeras evidencias indirectas de la existencia de materia oscura. Se tiene un modelo de velocidades en función de la distancia al centro de la galaxia. Los datos deben de cuadrar con este y a partir de ahí se puede estimar la masa del bulbo $M_b$, la del disco estelar $M_d$ y la del halo de materia oscura $M_h$. \\

Usando estimación bayesiana de parámetros, se obtuvo el siguiente fit:

\begin{figure}[H]
    \centering
    \includegraphics[scale=0.6]{galaxias.png}
    \caption{Resultado del fit mediante estimación bayesiana y los datos dados.}
\end{figure}
\end{document}
